\documentclass[10pt,a5paper]{book}
\usepackage[utf8]{inputenc}
\usepackage{amsmath}
\usepackage{amsfonts}
\usepackage{amssymb}
\usepackage{graphicx}
\title{Ithunn}
\author{Marcel Gietzmann-Sanders}
\begin{document}
\maketitle
\tableofcontents
\newpage
\chapter*{Preface}
\addcontentsline{toc}{chapter}{Preface}
The past two years of searching have taught me one thing - I want my life's work to be Project Fjorgyn. I want to be participant in building the technology and organizations required for Stewardship in the Anthropocene. Yet I find myself faced with a dilemma. 

At the moment I am trying to split my time between Project Fjorgyn and my work with Viasat. This, combined with protecting the things of value in my life - friendships, family, romance, stories, the great outdoors and the exercise that comes with it - is burning me out. I have spent the last few months a ball of stress as I try to fit all of this in. 

At the center of this stress is a simple fact - I cannot dedicate enough time to do either my work or Project Fjorgyn properly. I am half-assing too many things. 

For a while I believed I needed to do this for two reasons - moral and psychological. In terms of the latter I originally was unsure what kind of work and activities really made me happy and provided me flow. After having spent time trying a variety of different things I have settled on the fact that I \textit{love} building things. As such, the activities I engage in with work or Project Fjorgyn have converged. This then leaves the moral component as the present reason for the divide. 

Over the past few months I have begun to realize the importance of what our team does at Viasat. This may seem odd given the horrifying things one will find with a quick search for our reviews, but to me this only drives the point home. Viasat has been allowed to provide such poor service at such high prices largely due to a lack of competition. It is, in my mind, the only reason we've been able to survive while managing our business and providing for our customers so poorly. Looking toward the future it has become clear that we do not have the tools nor processes in place to prevent our own undoing. And much of the pain our customers experience is due to our inability to plan or optimize our network. In other words - we are poor stewards of our network. 

Two moral implications arise from these considerations. First, that if we undo ourselves, we will leave a single entity in the space - Starlink - which will only allow them to repeat our sins. As a result helping Viasat remain in business can only work to benefit those who depend on satellite links for connection. Second is that Viasat has a \textit{moral} obligation to do better by their customers. There \textit{are} people who need this service to stay connected, and plenty more for which it is an enormous value add. We truly could connect the world, but only if we can be good stewards of our network. 

Those then are the moral implications for Viasat. What of their relationship to me? Well the last five years of being at this company has put me in a peculiar position where I happen to be the only one with a clear plan and the wherewithal to make this happen. Amidst the chaos our team has begun to create a island of sanity that must be developed if we are to be good stewards of our network for our customers. While it is true that this moral implications are far lesser than those that Project Fjorgyn seeks to address, my relevance to these issues is incredibly strong and unique. 

Furthermore, pursuing this work is not an abandonment of Project Fjorgyn by any means. The skills required here are, in the abstract, exactly the same as that for Project Fjorgyn - I will learn how to build organizations that can enable stewardship at scale. Think of this as possible the best schooling and practical education I could possibly get. We already know with great certainty that when I do move over to Project Fjorgyn that I will largely be working as an entrepreneur, so being able to educate myself first with a trial run is invaluable. 

In addition, this work is clearly finite - I expect it to take a year or two at most to complete (and less time if I'm able to dedicate more of myself to it). Therefore, once I am through with it I can put it aside and dedicate myself fully to Project Fjorgyn. 

Finally, I cannot simply end my career with Viasat now. This work is the culmination of all I have learned, and all the context and connections that I have gathered over the last five years. To drop it now would feel like yet another aspect of my life that just ended up wasted. What would have been the point of the last five years?

I can't keep half-assing two things that mean so much to me and I certainly can't stay in a burned out state continuously for the next few years. It's time to double down on one and make it \textit{the} passion project. The project I'm working on right now with Viasat is literally a trial run of how to do stewardship at scale and would allow me to help fulfill a moral good that our team is so unique to. This is the culmination of the last five years of work! Let's see it through.
\linebreak

August 3\textsuperscript{rd}, 2022

\end{document}